\documentclass[letterpaper,10pt,onecolumn,draftclsnofoot]{IEEEtran}
\usepackage{times}

\usepackage[english]{babel}
\usepackage[margin=0.75in]{geometry}

\title{Object speed Tracking}
\author{Alex Bailey, Ben Wick, Dylan Washburne\\CS 461, Fall Term}

\begin{document}

\begin{titlepage}

\maketitle

\begin{abstract}
Using a stationary camera, with the intent of being mounted on a car, we are attempting to  detect objects and determine the speeds of those objects relative to the Observer (camera).
This will be done by having the camera recognize objects in space and determine their speeds based on the rate at which they travel through the frame.
If the camera is on a moving object, then we will need to either have a way for the system to measure it's own speed, likely with an accelerometer or connect to the object, if the object is measuring it's own speed.
To make this work, we will have to research the varieties of cameras available to use, as well as the API’s  they operate with.
We will also have to review the available computer vision algorithms and determine which is the most appropriate.
From this, we will determine the best camera to be used and from there create a object tracking program.
 
\end{abstract}

\end{titlepage}


%Beginning of introduction
\section{Introduction}
\subsection{Purpose}
The purpose of this document is to present a detailed description of the requirements for the "Video Radar" software.
This document is intended for the main use of the client, as well as the professor and the teacher assistants and will be proposed to the client for its approval.

\subsection{Scope}
Our system, the \_\_\_\_, will be able to identify a specific type of object, such as a person or a car, calculate the object's speed, then display the speed on the screen.
Our system will be beneficial over other systems in that it can function unmanned, will specify which target is being tracked, and will be able to track multiple objects.
\subsection{Definitions, acronyms, and abbreviations}
\begin{tabular}{|p{4cm}|p{12cm}|}
	\hline
	\textbf{Term} & \textbf{Definition} \\
	\hline
	API (Application program interface) & A particular set of rules and specifications that software programs can follow to communicate with each other. \\
	\hline
	User & Someone who is interacting with the software. \\
	\hline
	Object & The entity being tracked by the video feed.  \\
	\hline
	
\end{tabular}

\subsection{References}

\subsection{Overview}
The rest of the document contains two additional sections.
The first section is the overall description.
This section will describe the intended use of the software and give background.
The last section is the specific requirements section.
This section contains all the software requirements.

%Beginning of overall desciption
\section{Overall description}
\subsection{Product perspective}
Our system is a self-contained product. 
Our system's interface will consist of a window where the user can activate the camera and speed tracking.
There will be buttons to stop and start the camera. 
There will be a menu bar that will allow the user to specify aspects of the system.
Our systems window will consist of a small button row at the bottom for essential buttons, such as the start and stop buttons.
There will be a row of drop down menus at the top.
We will be using computer vision software in order to identify the type of object and track its location across the frame.
The system will have only one mode of operation, the on mode, where the system constantly processes the images from the camera and speed is displayed.
This mode requires no input from the user except to turn off.
\subsection{Product functions}
The software will be connected to a camera with a live video feed.
It will than be able to detect objects that are specified for the users needs.
The software will than be able to acquire the speed at which they are traveling.
This information will be stored into a table for the user to see.


\subsection{User characteristics}
This software is intended to be used by many different users.
The types of users are broken down into two categories: users who wish to track cars, users who wish to track people.
These users have different use for the system but the software should work the same way for both users.

Users who wish to track cars may use the software to detect the speed of a moving car on the road.
This means the user will set the stationary camera and point it in the direction of moving vehicles to obtain the speed.

Users who wish to track the speed of people may also use the software.
These users will follow the same procedure as the other users but instead, the software will detect the speed of people.

\subsection{Constraints}
This product will likely require a nontrivial amount of resources to perform its task at the constant interval we require.
As a result, our product will either need to come with a dedicated computer to do the processing, or it will have to interface with existing computers which we expect to be in the locations of use.

In accordance with federal laws, the camera is allowed to be recording anything described as "within plain sight".
The proper use of this falls on the user, and must be disclaimed before use.

The product's reliability on use can be described in a number of ways.
The camera should be recording at a constant and stable rate.
The objects in the scene should be properly identified with 70\% accuracy.
The tracking of an identified object's speed should be within 90\% of the objects actual speed.
The speed tracking should also be reliably given every 0.5 seconds, without any notable dips in rate of recording and processing.

While the video recorded is protected by federal laws, the information nonetheless must meet certain security standards.
If the video data is to be saved locally, it should also respect encryption of the system it is saved on.
Should the video be live streamed to a remote location, it must enter a stable connection to deliver to the intended viewer.

\subsection{Assumptions and dependencies}
We assume that if our systems camera needs to move that our system would be able to either handle the motion of the camera and still be able to track speeds or be mounted securely enough to minimize motion of the camera, allowing the system to track speeds.
We assume that the computer vision software will be able to recognize an object within enough time after entering the frame so that there will be enough time for the speed algorithm to calculate the speed.
%Beginning of specific requirments
\section{Specific requirements}

\end{document}