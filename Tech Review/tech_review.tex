\documentclass[letterpaper,10pt,onecolumn,draftclsnofoot]{IEEEtran}
\usepackage{times}

\usepackage[english]{babel}
\usepackage[margin=0.75in]{geometry}

\usepackage{graphicx}

\DeclareGraphicsExtensions{.pdf,.png,.jpg}

\title{Object Speed Tracking}
\author{Tech Review\\Alex Bailey, Ben Wick, Dylan Washburne\\CS 461, Fall Term}

\begin{document}

\begin{titlepage}

\maketitle

\begin{abstract}
While we could consider whether we would use a singular camera or multiple, there is really no debate here.
Stereoscopic cameras carry such an advantage for a project like this, we are making that decision.
 
\end{abstract}

\end{titlepage}

\tableofcontents
\newpage

\section{Introduction}

\section{Technologies}

\subsection{Live Data Feed} %Alex

Both images simultaneously
Both images plus depth information
Singular image with depth

\newpage
\subsection{Live Compression} %Dylan

No compression
Camera live compresses
Computer live compresses

\newpage
\subsection{Long-Term Storage}%Ben

1 feed+data, 
2 feeds no data to be re-calculated, 
1 feed data baked into video

\newpage
\subsection{Computer Vision Library} %Ben
The three options for Computer Vision (CV) libraries include OpenCV, LTI-Lib, and VXL.
Selecting a good CV library is essential for our project.
The goal of the CV library is to provide us with a large array of functions that we are able to utilize.
Every CV library we are looking at is open source and is free to use.
The library selected must be able to support our needs of being able to identify and track objects in real time through our live video feed.
It must have many available tools to simplify the identification and the tracking of objects.
The criteria that will be evaluated are languages available, features available, and performance.
The languages available is important because choosing a language we are already familiar with will give us an advantage.
We are hoping to use C/C++.
Feature available is also another very important criteria.
Specifically, we are looking for features that are able to simplify object tracking and detection.
Performance also plays a huge large factor in our choice.
We are looking for library that is as efficient as possible.
These libraries tend to be written in C/C++.

OpenCV is one of the most commonly used libraries for computer vision.
The OpenCV libraries are written in C/C++.
According to their website, they offer over 2500 algorithms [1].
The purpose of OpenCV is to offer a large number of function that the user is able to use to simplify difficult tasks.
This includes algorithms that are capable to identify and track objects, recognize faces, follow eye movements and many more [1].
OpenCV is used by many large companies like Google, Yahoo, Microsoft, Intel, IBM, Sony, Honda and Toyota [1]. 

LTI-Lib is also another Computer vision library that has many algorithms and features.
LTI-Lib's main goal is to provide an object oriented library in C++ [2].
It includes libraries for linear algebra, classification and clustering, image processing, as well as visualization and drawing tools [2].
There are a few research projects that utilize LTI-Lib including one at the University of Liege, Belgium where they do many things like machine learning, medical imaging, radar imaging, as well as sports video analysis.


VXL is another great computer vision library utilized by a lot of people.
Similar to OpenCV, the VXL libraries are written more in C++ [3].
The main libraries within VXL include numerics (VNL), imaging (VIL), geometry (VGL), and streaming I/O (VSL) [4].
They also have libraries for image processing, camera geometry, stereo, video manipulation, 3D imaging and many more [4].
VXL is written to be light, fast, and portable over many platforms [5].


\begin{center}
	\begin{tabular}{|c|c|c|c|}
		
		\hline
		\textbf{} & \textbf{Languages available} & \textbf{Features available} & \textbf{Performance (rank)} \\
		\hline
		OpenCV & C, C++, Java, Python & Many & 1 \\
		\hline
		LTI-lib & C, C++ & Many & 2 \\
		\hline
		VXL & C, C++ & Numerics, Imaging, geometry, streaming I/O & 3 \\
		\hline
		
	\end{tabular}
\end{center}

Based on the criteria needed for computer vision library, OpenCV has a large number of algorithms that we will be able to use as well as performs faster than the other libraries.
The speed performance test was done by Utkarsh Sinha.
The comparisons done were 2D DFT, resizing, optical flow, and neural net [6].
His findings show that OpenCV performs faster than both LTI-Lib and VXL with LTI-Lib coming in second [6].
Also, Based on reviews, OpenCV is just a lot more popular and widely used.
This is a huge positive because it means more references and documents will be available for use.
For our project I think OpenCV would be a lot more useful based on the criteria tested.

\newpage
\subsection{Computer Vision Underlying Algorithm} %Ben

Haar cascades
Background subraction

...
...

\newpage
\subsection{Synchronization} %Alex

two cameras, each sends images and computer works to compensate for any desyncronization
two cameras with global shutter, send back 2 images
2 cameras which do image splicing and send back spliced image to computer

\newpage
\subsection{UI Overlay} %Dylan

Open Broadcaster Software
VideoMeld
Camtasia

Goals for the selected software are to show the video stream as it is given, as well as placing boses around objects points sent from the backend's recognition software.

Criteria for each of these softwares include performance overhead and the ability to add more overlay components in real time.

\begin{tabular}{ l l }
  OBS Studio & OBS is a fairly common overlay software.  It has a stable level of performance and is able to pop in elements for the overlay at any time.  It is undetermined as of yet if this software would be able to make these additional elements go to dynamic positions on screen. \\ \hline
  VideoMeld & This seems to be from before the era of livestreams, and though it has a wide variety for overlay objects, I cannot find any way for it to support dynamically occuring objects in the overlay  \\ \hline
  Camtasia & This only works live through an extra plugin and it has significant overhead in the process.  Dynamic overlay objects do not seem to be supported. \\
\end{tabular}

Discussion

I choose OBS based on what is discussed above.

\newpage
\subsection{Existing Speed Formulas} %Dylan

make our own
...
...

\newpage
\subsection{Long-Term Compression} %Alex 
%Lossless
%Lossy
%No Compression

The goals for this piece of the project is to compress the video after it has been displayed to the user, to be kept for long term, should they be needed at a later date.
With our product, it is likely that the user will be leaving our product running for a significant amount of time, possibly hours.
When this happens, video files can become rather large.
This will become a problem for long term storage if our product is used often.
So, the answer to this problem is to use a compression codec.
This will reduce the size of the video as much as possible.
There are several factors to consider when looking at video compression codecs.
The first is whether or not it is lossless.
When compressing information, especially pictures and videos, the compression codecs will often save space by removing data, often in the form of merging pixels, leading to a lower resolution.
While a lossless codec is obviously prefered, there are not many lossless video compression methods available and they often have limited compression.
Second is the amount of compression, often expressed as a ratio.
This is the ration of the size of the original video file to the size of the compressed file.
This ration is typically better with a lossy, as the ability to reduce the quality of the video greatly reduces the size.
Third is the speed of the compression.
This is how long it takes for the codec to compress and decompress the video file.
While this isn't as important as the other 2, the less time that the user is kept waiting the better.

\newpage
\section{Conclusion}

\section{References}

[1]"ABOUT | OpenCV", Opencv.org, 2016. [Online]. Available: http://opencv.org/about.html. [Accessed: 16- Nov- 2016].

[2]"LTI-Lib", Ltilib.sourceforge.net, 2016. [Online]. Available: http://ltilib.sourceforge.net/doc/homepage/index.shtml. [Accessed: 16- Nov- 2016].

[3]"VXL: 1. Introduction", Public.kitware.com, 2016. [Online]. Available: http://public.kitware.com/vxl/doc/release/books/core/book_1.html. [Accessed: 16- Nov- 2016].

[4]"VXL - C++ Libraries for Computer Vision", Vxl.sourceforge.net, 2016. [Online]. Available: http://vxl.sourceforge.net/. [Accessed: 16- Nov- 2016].

[5]"Computer Recognition System For Detecting And Tracking Objects In 3D Environment", Csus-dspace.calstate.edu, 2016. [Online]. Available: http://csus-dspace.calstate.edu/bitstream/handle/10211.9/1249/Introduction%20to%20computer%20vision.pdf?sequence=2. [Accessed: 16- Nov- 2016].

[6]U. Sinha, "OpenCV vs VXL vs LTI: Performance Test - AI Shack - Tutorials for OpenCV, computer vision, deep learning, image processing, neural networks and artificial intelligence.", Aishack.in, 2016. [Online]. Available: http://aishack.in/tutorials/opencv-vs-vxl-vs-lti-performance-test/. [Accessed: 16- Nov- 2016].

%http://opencv.org/about.html [1]
%http://ltilib.sourceforge.net/doc/homepage/index.shtml [2]
%http://public.kitware.com/vxl/doc/release/books/core/book_1.html [3]
%http://vxl.sourceforge.net/ [4]
%http://csus-dspace.calstate.edu/bitstream/handle/10211.9/1249/Introduction%20to%20computer%20vision.pdf?sequence=2 [5]
%http://aishack.in/tutorials/opencv-vs-vxl-vs-lti-performance-test/[6]


%possible sections:  

%how the video feed is stored (1 feed+data, 2 feeds no data to be re-calculated, 1 feed data overlayed)
%connection speeds
%live compression
%long-term compression
%video to database storage
%mono- vs bi-focal
%computer vision
%speed formula
%legal accountability
%internal clocks/global shutters (syncronization)

\end{document}
