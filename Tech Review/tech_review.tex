\documentclass[letterpaper,10pt,onecolumn,draftclsnofoot]{IEEEtran}
\usepackage{times}

\usepackage[english]{babel}
\usepackage[margin=0.75in]{geometry}

\usepackage{graphicx}

\DeclareGraphicsExtensions{.pdf,.png,.jpg}

\title{Object Speed Tracking}
\author{Tech Review\\Alex Bailey, Ben Wick, Dylan Washburne\\CS 461, Fall Term}

\begin{document}

\begin{titlepage}

\maketitle

\begin{abstract}
While we could consider whether we would use a singular camera or multiple, there is really no debate here.
Stereoscopic cameras carry such an advantage for a project like this, we are making that decision.
 
\end{abstract}

\end{titlepage}

\section{Live Data Feed}

Both images simultaneously
Both images plus depth information
Singular image with depth

\section{Live Compression}

No compression
Camera live compresses
Computer live compresses

\section{Long-Term Storage}

1 feed+data, 
2 feeds no data to be re-calculated, 
1 feed data baked into video

\section{CV Software Package}

OpenCV
vlfeat
vxl

\section{CV Underlying Algorithm}

Harr cascades
...
...

\section{Synchronization}

two cameras, each sends images and computer works to compensate for any desyncronization
two cameras with global shutter, send back 2 images
2 cameras which do image splicing and send back spliced image to computer

\section{UI Overlay}

Open Broadcastcast Software
...
...

\section{Existing Speed Formulas}

make our own
...
...

\section{Long-Term Compression}

Lossless
Lossy
No Compression


%possible sections:  

%how the video feed is stored (1 feed+data, 2 feeds no data to be re-calculated, 1 feed data overlayed)
%connection speeds
%live compression
%long-term compression
%video to database storage
%mono- vs bi-focal
%computer vision
%speed formula
%legal accountability
%internal clocks/global shutters (syncronization)

\end{document}
