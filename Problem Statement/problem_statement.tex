\documentclass[letterpaper,10pt,onecolumn,draftclsnofoot]{IEEEtran}
\usepackage{times}

\usepackage[english]{babel}
\usepackage[margin=0.75in]{geometry}

\title{Object Velocity Tracking}
\author{Alex Bailey, Ben Wick, Dylan Washburne\\CS 461, Fall Term}

\begin{document}

\begin{titlepage}

\maketitle

\begin{abstract}
Using a camera mounted on a car, we are attempting to determine the speeds of other nearby vehicles.  This will be done by having the camera recognize nearby vehicles and determining their speeds relative to our own vehicle. To make this work, we will have to research the Kinect and how it processes images, aw well as how we can determine what in the field of view is a vehicle, and how fast it is moving relative to us.
\end{abstract}

\end{titlepage}

\section{Problem Definition}

There are many applications for tracking the speeds of moving objects.
Current methods include radar guns and laser scanners, however, speed tracking methods like this include various issues.
Primary among these issues is that radar and similar methods suffer heavily from poor weather conditions.
Methods like these are also only capable of tracking one object at a time.

%Distance is maybe an issue to look for?

\section{Proposed Solution}

We are going to use a camera to track the speed of a passing object.
The camera will be able to track passing objects and display their current speeds to the user.

\section{Performance Metrics}

Track the velocity of a nearby object, within 10\% of its actual speed.



%Applications somewhere maybe?


\end{document}